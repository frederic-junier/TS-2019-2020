
% Default to the notebook output style

    


% Inherit from the specified cell style.




    
\documentclass[11pt]{article}

    
    
    \usepackage[T1]{fontenc}
    % Nicer default font (+ math font) than Computer Modern for most use cases
    \usepackage{mathpazo}

    % Basic figure setup, for now with no caption control since it's done
    % automatically by Pandoc (which extracts ![](path) syntax from Markdown).
    \usepackage{graphicx}
    % We will generate all images so they have a width \maxwidth. This means
    % that they will get their normal width if they fit onto the page, but
    % are scaled down if they would overflow the margins.
    \makeatletter
    \def\maxwidth{\ifdim\Gin@nat@width>\linewidth\linewidth
    \else\Gin@nat@width\fi}
    \makeatother
    \let\Oldincludegraphics\includegraphics
    % Set max figure width to be 80% of text width, for now hardcoded.
    \renewcommand{\includegraphics}[1]{\Oldincludegraphics[width=.8\maxwidth]{#1}}
    % Ensure that by default, figures have no caption (until we provide a
    % proper Figure object with a Caption API and a way to capture that
    % in the conversion process - todo).
    \usepackage{caption}
    \DeclareCaptionLabelFormat{nolabel}{}
    \captionsetup{labelformat=nolabel}

    \usepackage{adjustbox} % Used to constrain images to a maximum size 
    \usepackage{xcolor} % Allow colors to be defined
    \usepackage{enumerate} % Needed for markdown enumerations to work
    \usepackage{geometry} % Used to adjust the document margins
    \usepackage{amsmath} % Equations
    \usepackage{amssymb} % Equations
    \usepackage{textcomp} % defines textquotesingle
    % Hack from http://tex.stackexchange.com/a/47451/13684:
    \AtBeginDocument{%
        \def\PYZsq{\textquotesingle}% Upright quotes in Pygmentized code
    }
    \usepackage{upquote} % Upright quotes for verbatim code
    \usepackage{eurosym} % defines \euro
    \usepackage[mathletters]{ucs} % Extended unicode (utf-8) support
    \usepackage[utf8x]{inputenc} % Allow utf-8 characters in the tex document
    \usepackage{fancyvrb} % verbatim replacement that allows latex
    \usepackage{grffile} % extends the file name processing of package graphics 
                         % to support a larger range 
    % The hyperref package gives us a pdf with properly built
    % internal navigation ('pdf bookmarks' for the table of contents,
    % internal cross-reference links, web links for URLs, etc.)
    \usepackage{hyperref}
    \usepackage{longtable} % longtable support required by pandoc >1.10
    \usepackage{booktabs}  % table support for pandoc > 1.12.2
    \usepackage[inline]{enumitem} % IRkernel/repr support (it uses the enumerate* environment)
    \usepackage[normalem]{ulem} % ulem is needed to support strikethroughs (\sout)
                                % normalem makes italics be italics, not underlines
    

    
    
    % Colors for the hyperref package
    \definecolor{urlcolor}{rgb}{0,.145,.698}
    \definecolor{linkcolor}{rgb}{.71,0.21,0.01}
    \definecolor{citecolor}{rgb}{.12,.54,.11}

    % ANSI colors
    \definecolor{ansi-black}{HTML}{3E424D}
    \definecolor{ansi-black-intense}{HTML}{282C36}
    \definecolor{ansi-red}{HTML}{E75C58}
    \definecolor{ansi-red-intense}{HTML}{B22B31}
    \definecolor{ansi-green}{HTML}{00A250}
    \definecolor{ansi-green-intense}{HTML}{007427}
    \definecolor{ansi-yellow}{HTML}{DDB62B}
    \definecolor{ansi-yellow-intense}{HTML}{B27D12}
    \definecolor{ansi-blue}{HTML}{208FFB}
    \definecolor{ansi-blue-intense}{HTML}{0065CA}
    \definecolor{ansi-magenta}{HTML}{D160C4}
    \definecolor{ansi-magenta-intense}{HTML}{A03196}
    \definecolor{ansi-cyan}{HTML}{60C6C8}
    \definecolor{ansi-cyan-intense}{HTML}{258F8F}
    \definecolor{ansi-white}{HTML}{C5C1B4}
    \definecolor{ansi-white-intense}{HTML}{A1A6B2}

    % commands and environments needed by pandoc snippets
    % extracted from the output of `pandoc -s`
    \providecommand{\tightlist}{%
      \setlength{\itemsep}{0pt}\setlength{\parskip}{0pt}}
    \DefineVerbatimEnvironment{Highlighting}{Verbatim}{commandchars=\\\{\}}
    % Add ',fontsize=\small' for more characters per line
    \newenvironment{Shaded}{}{}
    \newcommand{\KeywordTok}[1]{\textcolor[rgb]{0.00,0.44,0.13}{\textbf{{#1}}}}
    \newcommand{\DataTypeTok}[1]{\textcolor[rgb]{0.56,0.13,0.00}{{#1}}}
    \newcommand{\DecValTok}[1]{\textcolor[rgb]{0.25,0.63,0.44}{{#1}}}
    \newcommand{\BaseNTok}[1]{\textcolor[rgb]{0.25,0.63,0.44}{{#1}}}
    \newcommand{\FloatTok}[1]{\textcolor[rgb]{0.25,0.63,0.44}{{#1}}}
    \newcommand{\CharTok}[1]{\textcolor[rgb]{0.25,0.44,0.63}{{#1}}}
    \newcommand{\StringTok}[1]{\textcolor[rgb]{0.25,0.44,0.63}{{#1}}}
    \newcommand{\CommentTok}[1]{\textcolor[rgb]{0.38,0.63,0.69}{\textit{{#1}}}}
    \newcommand{\OtherTok}[1]{\textcolor[rgb]{0.00,0.44,0.13}{{#1}}}
    \newcommand{\AlertTok}[1]{\textcolor[rgb]{1.00,0.00,0.00}{\textbf{{#1}}}}
    \newcommand{\FunctionTok}[1]{\textcolor[rgb]{0.02,0.16,0.49}{{#1}}}
    \newcommand{\RegionMarkerTok}[1]{{#1}}
    \newcommand{\ErrorTok}[1]{\textcolor[rgb]{1.00,0.00,0.00}{\textbf{{#1}}}}
    \newcommand{\NormalTok}[1]{{#1}}
    
    % Additional commands for more recent versions of Pandoc
    \newcommand{\ConstantTok}[1]{\textcolor[rgb]{0.53,0.00,0.00}{{#1}}}
    \newcommand{\SpecialCharTok}[1]{\textcolor[rgb]{0.25,0.44,0.63}{{#1}}}
    \newcommand{\VerbatimStringTok}[1]{\textcolor[rgb]{0.25,0.44,0.63}{{#1}}}
    \newcommand{\SpecialStringTok}[1]{\textcolor[rgb]{0.73,0.40,0.53}{{#1}}}
    \newcommand{\ImportTok}[1]{{#1}}
    \newcommand{\DocumentationTok}[1]{\textcolor[rgb]{0.73,0.13,0.13}{\textit{{#1}}}}
    \newcommand{\AnnotationTok}[1]{\textcolor[rgb]{0.38,0.63,0.69}{\textbf{\textit{{#1}}}}}
    \newcommand{\CommentVarTok}[1]{\textcolor[rgb]{0.38,0.63,0.69}{\textbf{\textit{{#1}}}}}
    \newcommand{\VariableTok}[1]{\textcolor[rgb]{0.10,0.09,0.49}{{#1}}}
    \newcommand{\ControlFlowTok}[1]{\textcolor[rgb]{0.00,0.44,0.13}{\textbf{{#1}}}}
    \newcommand{\OperatorTok}[1]{\textcolor[rgb]{0.40,0.40,0.40}{{#1}}}
    \newcommand{\BuiltInTok}[1]{{#1}}
    \newcommand{\ExtensionTok}[1]{{#1}}
    \newcommand{\PreprocessorTok}[1]{\textcolor[rgb]{0.74,0.48,0.00}{{#1}}}
    \newcommand{\AttributeTok}[1]{\textcolor[rgb]{0.49,0.56,0.16}{{#1}}}
    \newcommand{\InformationTok}[1]{\textcolor[rgb]{0.38,0.63,0.69}{\textbf{\textit{{#1}}}}}
    \newcommand{\WarningTok}[1]{\textcolor[rgb]{0.38,0.63,0.69}{\textbf{\textit{{#1}}}}}
    
    
    % Define a nice break command that doesn't care if a line doesn't already
    % exist.
    \def\br{\hspace*{\fill} \\* }
    % Math Jax compatability definitions
    \def\gt{>}
    \def\lt{<}
    % Document parameters
    \title{Cours-D?rivation-Dichotomie-Exemple12}
    
    
    

    % Pygments definitions
    
\makeatletter
\def\PY@reset{\let\PY@it=\relax \let\PY@bf=\relax%
    \let\PY@ul=\relax \let\PY@tc=\relax%
    \let\PY@bc=\relax \let\PY@ff=\relax}
\def\PY@tok#1{\csname PY@tok@#1\endcsname}
\def\PY@toks#1+{\ifx\relax#1\empty\else%
    \PY@tok{#1}\expandafter\PY@toks\fi}
\def\PY@do#1{\PY@bc{\PY@tc{\PY@ul{%
    \PY@it{\PY@bf{\PY@ff{#1}}}}}}}
\def\PY#1#2{\PY@reset\PY@toks#1+\relax+\PY@do{#2}}

\expandafter\def\csname PY@tok@w\endcsname{\def\PY@tc##1{\textcolor[rgb]{0.73,0.73,0.73}{##1}}}
\expandafter\def\csname PY@tok@c\endcsname{\let\PY@it=\textit\def\PY@tc##1{\textcolor[rgb]{0.25,0.50,0.50}{##1}}}
\expandafter\def\csname PY@tok@cp\endcsname{\def\PY@tc##1{\textcolor[rgb]{0.74,0.48,0.00}{##1}}}
\expandafter\def\csname PY@tok@k\endcsname{\let\PY@bf=\textbf\def\PY@tc##1{\textcolor[rgb]{0.00,0.50,0.00}{##1}}}
\expandafter\def\csname PY@tok@kp\endcsname{\def\PY@tc##1{\textcolor[rgb]{0.00,0.50,0.00}{##1}}}
\expandafter\def\csname PY@tok@kt\endcsname{\def\PY@tc##1{\textcolor[rgb]{0.69,0.00,0.25}{##1}}}
\expandafter\def\csname PY@tok@o\endcsname{\def\PY@tc##1{\textcolor[rgb]{0.40,0.40,0.40}{##1}}}
\expandafter\def\csname PY@tok@ow\endcsname{\let\PY@bf=\textbf\def\PY@tc##1{\textcolor[rgb]{0.67,0.13,1.00}{##1}}}
\expandafter\def\csname PY@tok@nb\endcsname{\def\PY@tc##1{\textcolor[rgb]{0.00,0.50,0.00}{##1}}}
\expandafter\def\csname PY@tok@nf\endcsname{\def\PY@tc##1{\textcolor[rgb]{0.00,0.00,1.00}{##1}}}
\expandafter\def\csname PY@tok@nc\endcsname{\let\PY@bf=\textbf\def\PY@tc##1{\textcolor[rgb]{0.00,0.00,1.00}{##1}}}
\expandafter\def\csname PY@tok@nn\endcsname{\let\PY@bf=\textbf\def\PY@tc##1{\textcolor[rgb]{0.00,0.00,1.00}{##1}}}
\expandafter\def\csname PY@tok@ne\endcsname{\let\PY@bf=\textbf\def\PY@tc##1{\textcolor[rgb]{0.82,0.25,0.23}{##1}}}
\expandafter\def\csname PY@tok@nv\endcsname{\def\PY@tc##1{\textcolor[rgb]{0.10,0.09,0.49}{##1}}}
\expandafter\def\csname PY@tok@no\endcsname{\def\PY@tc##1{\textcolor[rgb]{0.53,0.00,0.00}{##1}}}
\expandafter\def\csname PY@tok@nl\endcsname{\def\PY@tc##1{\textcolor[rgb]{0.63,0.63,0.00}{##1}}}
\expandafter\def\csname PY@tok@ni\endcsname{\let\PY@bf=\textbf\def\PY@tc##1{\textcolor[rgb]{0.60,0.60,0.60}{##1}}}
\expandafter\def\csname PY@tok@na\endcsname{\def\PY@tc##1{\textcolor[rgb]{0.49,0.56,0.16}{##1}}}
\expandafter\def\csname PY@tok@nt\endcsname{\let\PY@bf=\textbf\def\PY@tc##1{\textcolor[rgb]{0.00,0.50,0.00}{##1}}}
\expandafter\def\csname PY@tok@nd\endcsname{\def\PY@tc##1{\textcolor[rgb]{0.67,0.13,1.00}{##1}}}
\expandafter\def\csname PY@tok@s\endcsname{\def\PY@tc##1{\textcolor[rgb]{0.73,0.13,0.13}{##1}}}
\expandafter\def\csname PY@tok@sd\endcsname{\let\PY@it=\textit\def\PY@tc##1{\textcolor[rgb]{0.73,0.13,0.13}{##1}}}
\expandafter\def\csname PY@tok@si\endcsname{\let\PY@bf=\textbf\def\PY@tc##1{\textcolor[rgb]{0.73,0.40,0.53}{##1}}}
\expandafter\def\csname PY@tok@se\endcsname{\let\PY@bf=\textbf\def\PY@tc##1{\textcolor[rgb]{0.73,0.40,0.13}{##1}}}
\expandafter\def\csname PY@tok@sr\endcsname{\def\PY@tc##1{\textcolor[rgb]{0.73,0.40,0.53}{##1}}}
\expandafter\def\csname PY@tok@ss\endcsname{\def\PY@tc##1{\textcolor[rgb]{0.10,0.09,0.49}{##1}}}
\expandafter\def\csname PY@tok@sx\endcsname{\def\PY@tc##1{\textcolor[rgb]{0.00,0.50,0.00}{##1}}}
\expandafter\def\csname PY@tok@m\endcsname{\def\PY@tc##1{\textcolor[rgb]{0.40,0.40,0.40}{##1}}}
\expandafter\def\csname PY@tok@gh\endcsname{\let\PY@bf=\textbf\def\PY@tc##1{\textcolor[rgb]{0.00,0.00,0.50}{##1}}}
\expandafter\def\csname PY@tok@gu\endcsname{\let\PY@bf=\textbf\def\PY@tc##1{\textcolor[rgb]{0.50,0.00,0.50}{##1}}}
\expandafter\def\csname PY@tok@gd\endcsname{\def\PY@tc##1{\textcolor[rgb]{0.63,0.00,0.00}{##1}}}
\expandafter\def\csname PY@tok@gi\endcsname{\def\PY@tc##1{\textcolor[rgb]{0.00,0.63,0.00}{##1}}}
\expandafter\def\csname PY@tok@gr\endcsname{\def\PY@tc##1{\textcolor[rgb]{1.00,0.00,0.00}{##1}}}
\expandafter\def\csname PY@tok@ge\endcsname{\let\PY@it=\textit}
\expandafter\def\csname PY@tok@gs\endcsname{\let\PY@bf=\textbf}
\expandafter\def\csname PY@tok@gp\endcsname{\let\PY@bf=\textbf\def\PY@tc##1{\textcolor[rgb]{0.00,0.00,0.50}{##1}}}
\expandafter\def\csname PY@tok@go\endcsname{\def\PY@tc##1{\textcolor[rgb]{0.53,0.53,0.53}{##1}}}
\expandafter\def\csname PY@tok@gt\endcsname{\def\PY@tc##1{\textcolor[rgb]{0.00,0.27,0.87}{##1}}}
\expandafter\def\csname PY@tok@err\endcsname{\def\PY@bc##1{\setlength{\fboxsep}{0pt}\fcolorbox[rgb]{1.00,0.00,0.00}{1,1,1}{\strut ##1}}}
\expandafter\def\csname PY@tok@kc\endcsname{\let\PY@bf=\textbf\def\PY@tc##1{\textcolor[rgb]{0.00,0.50,0.00}{##1}}}
\expandafter\def\csname PY@tok@kd\endcsname{\let\PY@bf=\textbf\def\PY@tc##1{\textcolor[rgb]{0.00,0.50,0.00}{##1}}}
\expandafter\def\csname PY@tok@kn\endcsname{\let\PY@bf=\textbf\def\PY@tc##1{\textcolor[rgb]{0.00,0.50,0.00}{##1}}}
\expandafter\def\csname PY@tok@kr\endcsname{\let\PY@bf=\textbf\def\PY@tc##1{\textcolor[rgb]{0.00,0.50,0.00}{##1}}}
\expandafter\def\csname PY@tok@bp\endcsname{\def\PY@tc##1{\textcolor[rgb]{0.00,0.50,0.00}{##1}}}
\expandafter\def\csname PY@tok@fm\endcsname{\def\PY@tc##1{\textcolor[rgb]{0.00,0.00,1.00}{##1}}}
\expandafter\def\csname PY@tok@vc\endcsname{\def\PY@tc##1{\textcolor[rgb]{0.10,0.09,0.49}{##1}}}
\expandafter\def\csname PY@tok@vg\endcsname{\def\PY@tc##1{\textcolor[rgb]{0.10,0.09,0.49}{##1}}}
\expandafter\def\csname PY@tok@vi\endcsname{\def\PY@tc##1{\textcolor[rgb]{0.10,0.09,0.49}{##1}}}
\expandafter\def\csname PY@tok@vm\endcsname{\def\PY@tc##1{\textcolor[rgb]{0.10,0.09,0.49}{##1}}}
\expandafter\def\csname PY@tok@sa\endcsname{\def\PY@tc##1{\textcolor[rgb]{0.73,0.13,0.13}{##1}}}
\expandafter\def\csname PY@tok@sb\endcsname{\def\PY@tc##1{\textcolor[rgb]{0.73,0.13,0.13}{##1}}}
\expandafter\def\csname PY@tok@sc\endcsname{\def\PY@tc##1{\textcolor[rgb]{0.73,0.13,0.13}{##1}}}
\expandafter\def\csname PY@tok@dl\endcsname{\def\PY@tc##1{\textcolor[rgb]{0.73,0.13,0.13}{##1}}}
\expandafter\def\csname PY@tok@s2\endcsname{\def\PY@tc##1{\textcolor[rgb]{0.73,0.13,0.13}{##1}}}
\expandafter\def\csname PY@tok@sh\endcsname{\def\PY@tc##1{\textcolor[rgb]{0.73,0.13,0.13}{##1}}}
\expandafter\def\csname PY@tok@s1\endcsname{\def\PY@tc##1{\textcolor[rgb]{0.73,0.13,0.13}{##1}}}
\expandafter\def\csname PY@tok@mb\endcsname{\def\PY@tc##1{\textcolor[rgb]{0.40,0.40,0.40}{##1}}}
\expandafter\def\csname PY@tok@mf\endcsname{\def\PY@tc##1{\textcolor[rgb]{0.40,0.40,0.40}{##1}}}
\expandafter\def\csname PY@tok@mh\endcsname{\def\PY@tc##1{\textcolor[rgb]{0.40,0.40,0.40}{##1}}}
\expandafter\def\csname PY@tok@mi\endcsname{\def\PY@tc##1{\textcolor[rgb]{0.40,0.40,0.40}{##1}}}
\expandafter\def\csname PY@tok@il\endcsname{\def\PY@tc##1{\textcolor[rgb]{0.40,0.40,0.40}{##1}}}
\expandafter\def\csname PY@tok@mo\endcsname{\def\PY@tc##1{\textcolor[rgb]{0.40,0.40,0.40}{##1}}}
\expandafter\def\csname PY@tok@ch\endcsname{\let\PY@it=\textit\def\PY@tc##1{\textcolor[rgb]{0.25,0.50,0.50}{##1}}}
\expandafter\def\csname PY@tok@cm\endcsname{\let\PY@it=\textit\def\PY@tc##1{\textcolor[rgb]{0.25,0.50,0.50}{##1}}}
\expandafter\def\csname PY@tok@cpf\endcsname{\let\PY@it=\textit\def\PY@tc##1{\textcolor[rgb]{0.25,0.50,0.50}{##1}}}
\expandafter\def\csname PY@tok@c1\endcsname{\let\PY@it=\textit\def\PY@tc##1{\textcolor[rgb]{0.25,0.50,0.50}{##1}}}
\expandafter\def\csname PY@tok@cs\endcsname{\let\PY@it=\textit\def\PY@tc##1{\textcolor[rgb]{0.25,0.50,0.50}{##1}}}

\def\PYZbs{\char`\\}
\def\PYZus{\char`\_}
\def\PYZob{\char`\{}
\def\PYZcb{\char`\}}
\def\PYZca{\char`\^}
\def\PYZam{\char`\&}
\def\PYZlt{\char`\<}
\def\PYZgt{\char`\>}
\def\PYZsh{\char`\#}
\def\PYZpc{\char`\%}
\def\PYZdl{\char`\$}
\def\PYZhy{\char`\-}
\def\PYZsq{\char`\'}
\def\PYZdq{\char`\"}
\def\PYZti{\char`\~}
% for compatibility with earlier versions
\def\PYZat{@}
\def\PYZlb{[}
\def\PYZrb{]}
\makeatother


    % Exact colors from NB
    \definecolor{incolor}{rgb}{0.0, 0.0, 0.5}
    \definecolor{outcolor}{rgb}{0.545, 0.0, 0.0}



    
    % Prevent overflowing lines due to hard-to-break entities
    \sloppy 
    % Setup hyperref package
    \hypersetup{
      breaklinks=true,  % so long urls are correctly broken across lines
      colorlinks=true,
      urlcolor=urlcolor,
      linkcolor=linkcolor,
      citecolor=citecolor,
      }
    % Slightly bigger margins than the latex defaults
    
    \geometry{verbose,tmargin=1in,bmargin=1in,lmargin=1in,rmargin=1in}
    
    

    \begin{document}
    
    
    \maketitle
    
    

    
    \subsection{Import des bibliothèques
Python}\label{import-des-bibliothuxe8ques-python}

    \begin{Verbatim}[commandchars=\\\{\}]
{\color{incolor}In [{\color{incolor}1}]:} \PY{k+kn}{import} \PY{n+nn}{numpy} \PY{k}{as} \PY{n+nn}{np}                \PY{c+c1}{\PYZsh{}pour disposer des tableaux de type array}
        \PY{k+kn}{import} \PY{n+nn}{matplotlib}\PY{n+nn}{.}\PY{n+nn}{pyplot} \PY{k}{as} \PY{n+nn}{plt}   \PY{c+c1}{\PYZsh{}pour les graphiques}
\end{Verbatim}


    \begin{Verbatim}[commandchars=\\\{\}]
{\color{incolor}In [{\color{incolor}3}]:} \PY{o}{\PYZpc{}}\PY{k}{matplotlib} inline
        \PY{c+c1}{\PYZsh{}pour l\PYZsq{}affichage des graphiques dans la page et non pas dans une fenetre pop up}
\end{Verbatim}


    \begin{Verbatim}[commandchars=\\\{\}]
{\color{incolor}In [{\color{incolor}4}]:} \PY{k+kn}{import} \PY{n+nn}{operator}                   \PY{c+c1}{\PYZsh{}pour utiliser les opérateurs de base sous forme de fonctions}
\end{Verbatim}


    \begin{Verbatim}[commandchars=\\\{\}]
{\color{incolor}In [{\color{incolor}5}]:} \PY{k+kn}{from} \PY{n+nn}{sympy} \PY{k}{import} \PY{o}{*}               \PY{c+c1}{\PYZsh{}pour le calcul formel}
        \PY{n}{init\PYZus{}printing}\PY{p}{(}\PY{p}{)}
        \PY{n}{t} \PY{o}{=} \PY{n}{symbols}\PY{p}{(}\PY{l+s+s1}{\PYZsq{}}\PY{l+s+s1}{t}\PY{l+s+s1}{\PYZsq{}}\PY{p}{)}
\end{Verbatim}


    \begin{Verbatim}[commandchars=\\\{\}]
{\color{incolor}In [{\color{incolor}10}]:} \PY{k}{def} \PY{n+nf}{dérivée}\PY{p}{(}\PY{n}{exp}\PY{p}{,} \PY{n}{t}\PY{p}{)}\PY{p}{:}
             \PY{k}{return} \PY{n}{diff}\PY{p}{(}\PY{n}{exp}\PY{p}{,}\PY{n}{t}\PY{p}{)}
         
         \PY{k}{def} \PY{n+nf}{simplifier}\PY{p}{(}\PY{n}{exp}\PY{p}{)}\PY{p}{:}
             \PY{k}{return} \PY{n}{simplify}\PY{p}{(}\PY{n}{exp}\PY{p}{)}
         
         \PY{k}{def} \PY{n+nf}{factoriser}\PY{p}{(}\PY{n}{exp}\PY{p}{)}\PY{p}{:}
             \PY{k}{return} \PY{n}{factor}\PY{p}{(}\PY{n}{exp}\PY{p}{)}
\end{Verbatim}


    \subsection{\texorpdfstring{Résolution approchée de \(f(x)=0\) par
dichotomie, exemple 12 du
cours}{Résolution approchée de f(x)=0 par dichotomie, exemple 12 du cours}}\label{ruxe9solution-approchuxe9e-de-fx0-par-dichotomie-exemple-12-du-cours}

    Soit \(f\) la fonction définie sur \(\mathbb{R}\) par
\(f(x)=40x^3-561x^2+1917x-200\) definie sur \(\mathbb{R}\) et dérivable
sur \(\mathbb{R}\) .

    \subsubsection{Question 1 : Calcul de
dérivée}\label{question-1-calcul-de-duxe9rivuxe9e}

    \begin{Verbatim}[commandchars=\\\{\}]
{\color{incolor}In [{\color{incolor}7}]:} \PY{c+c1}{\PYZsh{}expression de f(x)}
        \PY{n}{fexp} \PY{o}{=} \PY{l+m+mi}{40} \PY{o}{*} \PY{n}{t}\PY{o}{*}\PY{o}{*}\PY{l+m+mi}{3} \PY{o}{\PYZhy{}} \PY{l+m+mi}{561}\PY{o}{*}\PY{n}{t}\PY{o}{*}\PY{o}{*}\PY{l+m+mi}{2} \PY{o}{+} \PY{l+m+mi}{1917} \PY{o}{*} \PY{n}{t} \PY{o}{\PYZhy{}} \PY{l+m+mi}{200}
        \PY{n}{fexp}
\end{Verbatim}

\texttt{\color{outcolor}Out[{\color{outcolor}7}]:}
    
    $\displaystyle 40 t^{3} - 561 t^{2} + 1917 t - 200$

    

    \begin{Verbatim}[commandchars=\\\{\}]
{\color{incolor}In [{\color{incolor}8}]:} \PY{c+c1}{\PYZsh{}expression de f\PYZsq{}(x)}
        \PY{n}{fprimexp} \PY{o}{=} \PY{n}{dérivée}\PY{p}{(}\PY{n}{fexp}\PY{p}{,} \PY{n}{t}\PY{p}{)}
        \PY{n}{fprimexp}
\end{Verbatim}

\texttt{\color{outcolor}Out[{\color{outcolor}8}]:}
    
    $\displaystyle 120 t^{2} - 1122 t + 1917$

    

    \begin{Verbatim}[commandchars=\\\{\}]
{\color{incolor}In [{\color{incolor}11}]:} \PY{n}{factoriser}\PY{p}{(}\PY{n}{fprimexp}\PY{p}{)}
\end{Verbatim}

\texttt{\color{outcolor}Out[{\color{outcolor}11}]:}
    
    $\displaystyle 3 \left(4 t - 9\right) \left(10 t - 71\right)$

    

    \subsubsection{\texorpdfstring{Questions 2 et 3 Etude des variations de
\(f\)}{Questions 2 et 3 Etude des variations de f}}\label{questions-2-et-3-etude-des-variations-de-f}

    \begin{Verbatim}[commandchars=\\\{\}]
{\color{incolor}In [{\color{incolor}12}]:} \PY{n}{f} \PY{o}{=} \PY{n}{lambdify}\PY{p}{(}\PY{n}{t}\PY{p}{,} \PY{n}{fexp}\PY{p}{,}\PY{l+s+s2}{\PYZdq{}}\PY{l+s+s2}{numpy}\PY{l+s+s2}{\PYZdq{}}\PY{p}{)}
         \PY{n}{fprim} \PY{o}{=} \PY{n}{lambdify}\PY{p}{(}\PY{n}{t}\PY{p}{,} \PY{n}{fprimexp}\PY{p}{,}\PY{l+s+s2}{\PYZdq{}}\PY{l+s+s2}{numpy}\PY{l+s+s2}{\PYZdq{}}\PY{p}{)}
\end{Verbatim}


    \begin{Verbatim}[commandchars=\\\{\}]
{\color{incolor}In [{\color{incolor}13}]:} \PY{c+c1}{\PYZsh{}tracé des  courbes  de f et f\PYZsq{}}
         \PY{c+c1}{\PYZsh{}Message d\PYZsq{}erreur pour f\PYZsq{} pour le point d\PYZsq{}abscisse 0 (division par 0)}
         \PY{n}{xmin}\PY{p}{,} \PY{n}{xmax}\PY{p}{,} \PY{n}{ymin}\PY{p}{,} \PY{n}{ymax} \PY{o}{=} \PY{l+m+mi}{0}\PY{p}{,} \PY{l+m+mi}{9}\PY{p}{,} \PY{o}{\PYZhy{}}\PY{l+m+mi}{600} \PY{p}{,} \PY{l+m+mi}{1800}
         \PY{n}{plt}\PY{o}{.}\PY{n}{axis}\PY{p}{(}\PY{p}{[}\PY{n}{xmin}\PY{p}{,} \PY{n}{xmax}\PY{p}{,} \PY{n}{ymin}\PY{p}{,} \PY{n}{ymax}\PY{p}{]}\PY{p}{)}
         \PY{n}{tx} \PY{o}{=} \PY{n}{np}\PY{o}{.}\PY{n}{linspace}\PY{p}{(}\PY{n}{xmin}\PY{p}{,} \PY{n}{xmax}\PY{p}{,} \PY{l+m+mi}{1001}\PY{p}{)}
         \PY{n}{ty} \PY{o}{=} \PY{n}{f}\PY{p}{(}\PY{n}{tx}\PY{p}{)}
         \PY{n}{tz} \PY{o}{=} \PY{n}{fprim}\PY{p}{(}\PY{n}{tx}\PY{p}{)}
         \PY{n}{plt}\PY{o}{.}\PY{n}{axhline}\PY{p}{(}\PY{n}{color}\PY{o}{=}\PY{l+s+s1}{\PYZsq{}}\PY{l+s+s1}{blue}\PY{l+s+s1}{\PYZsq{}}\PY{p}{)}
         \PY{n}{plt}\PY{o}{.}\PY{n}{axvline}\PY{p}{(}\PY{n}{color}\PY{o}{=}\PY{l+s+s1}{\PYZsq{}}\PY{l+s+s1}{blue}\PY{l+s+s1}{\PYZsq{}}\PY{p}{)}
         \PY{n}{plt}\PY{o}{.}\PY{n}{grid}\PY{p}{(}\PY{k+kc}{True}\PY{p}{)}
         \PY{n}{plt}\PY{o}{.}\PY{n}{plot}\PY{p}{(}\PY{n}{tx}\PY{p}{,} \PY{n}{ty}\PY{p}{,} \PY{n}{linestyle}\PY{o}{=}\PY{l+s+s1}{\PYZsq{}}\PY{l+s+s1}{\PYZhy{}}\PY{l+s+s1}{\PYZsq{}}\PY{p}{,} \PY{n}{linewidth}\PY{o}{=}\PY{l+m+mi}{2}\PY{p}{,} \PY{n}{color}\PY{o}{=}\PY{l+s+s1}{\PYZsq{}}\PY{l+s+s1}{red}\PY{l+s+s1}{\PYZsq{}}\PY{p}{,} \PY{n}{label}\PY{o}{=}\PY{l+s+sa}{r}\PY{l+s+s1}{\PYZsq{}}\PY{l+s+s1}{\PYZdl{}y=f(x)\PYZdl{}}\PY{l+s+s1}{\PYZsq{}}\PY{p}{)}
         \PY{n}{plt}\PY{o}{.}\PY{n}{plot}\PY{p}{(}\PY{n}{tx}\PY{p}{,} \PY{n}{tz}\PY{p}{,} \PY{n}{linestyle}\PY{o}{=}\PY{l+s+s1}{\PYZsq{}}\PY{l+s+s1}{\PYZhy{}}\PY{l+s+s1}{\PYZsq{}}\PY{p}{,} \PY{n}{linewidth}\PY{o}{=}\PY{l+m+mi}{1}\PY{p}{,} \PY{n}{color}\PY{o}{=}\PY{l+s+s1}{\PYZsq{}}\PY{l+s+s1}{green}\PY{l+s+s1}{\PYZsq{}}\PY{p}{,} \PY{n}{label}\PY{o}{=}\PY{l+s+sa}{r}\PY{l+s+s2}{\PYZdq{}}\PY{l+s+s2}{\PYZdl{}y=f}\PY{l+s+s2}{\PYZsq{}}\PY{l+s+s2}{(x)\PYZdl{}}\PY{l+s+s2}{\PYZdq{}}\PY{p}{)}
         \PY{n}{plt}\PY{o}{.}\PY{n}{legend}\PY{p}{(}\PY{n}{loc}\PY{o}{=}\PY{l+s+s1}{\PYZsq{}}\PY{l+s+s1}{lower right}\PY{l+s+s1}{\PYZsq{}}\PY{p}{)}
                  
\end{Verbatim}


\begin{Verbatim}[commandchars=\\\{\}]
{\color{outcolor}Out[{\color{outcolor}13}]:} <matplotlib.legend.Legend at 0x7f8c4f7c5390>
\end{Verbatim}
            
    \begin{center}
    \adjustimage{max size={0.9\linewidth}{0.9\paperheight}}{output_14_1.png}
    \end{center}
    { \hspace*{\fill} \\}
    
    \subsubsection{\texorpdfstring{Question 4 Existence de solutions de
l'équation \(f(x)=0\) sur
\([8;9]\)}{Question 4 Existence de solutions de l'équation f(x)=0 sur {[}8;9{]}}}\label{question-4-existence-de-solutions-de-luxe9quation-fx0-sur-89}

    \begin{itemize}
\tightlist
\item
  \(f:x \mapsto 40x^3-561x^2+1917x-200\) est dérivable donc continue sur
  \([8;9]\)
\item
  \(f(8)<0\) et \(f(9)>0\)
\item
  \(f\) est strictement croissante sur \([8;9]\)
\end{itemize}

D'après un corollaire du théorème des valeurs intermédiaires, l'équation
\(f(x)=0\) possède donc une unique solution \(\alpha\) dans l'intervalle
\([8;9]\)

    \subsection{Résolution approchée par
balayage}\label{ruxe9solution-approchuxe9e-par-balayage}

    \begin{Verbatim}[commandchars=\\\{\}]
{\color{incolor}In [{\color{incolor}15}]:} \PY{k}{def} \PY{n+nf}{balayage}\PY{p}{(}\PY{n}{g}\PY{p}{,} \PY{n}{a}\PY{p}{,} \PY{n}{b}\PY{p}{,} \PY{n}{pas}\PY{p}{,} \PY{n}{k}\PY{p}{)}\PY{p}{:}
             \PY{l+s+sd}{\PYZdq{}\PYZdq{}\PYZdq{}Retourne un intervalle d\PYZsq{}amplitude pas encadrant l\PYZsq{}unique solution de g(x)= k}
         \PY{l+s+sd}{    dans l\PYZsq{}intervalle [a,b]\PYZdq{}\PYZdq{}\PYZdq{}}
             \PY{k}{if} \PY{n}{g}\PY{p}{(}\PY{n}{a}\PY{p}{)} \PY{o}{\PYZlt{}} \PY{n}{k}\PY{p}{:}
                 \PY{n}{comparaison} \PY{o}{=} \PY{k}{lambda} \PY{n}{u}\PY{p}{,} \PY{n}{v} \PY{p}{:} \PY{n}{operator}\PY{o}{.}\PY{n}{lt}\PY{p}{(}\PY{n}{u}\PY{p}{,}\PY{n}{v}\PY{p}{)}
             \PY{k}{else}\PY{p}{:}
                 \PY{n}{comparaison} \PY{o}{=} \PY{k}{lambda} \PY{n}{u}\PY{p}{,} \PY{n}{v} \PY{p}{:} \PY{n}{operator}\PY{o}{.}\PY{n}{gt}\PY{p}{(}\PY{n}{u}\PY{p}{,}\PY{n}{v}\PY{p}{)}
             \PY{n}{x} \PY{o}{=} \PY{n}{a}
             \PY{c+c1}{\PYZsh{}en\PYZhy{}tete du tableau}
             \PY{n+nb}{print}\PY{p}{(}\PY{l+s+s1}{\PYZsq{}}\PY{l+s+s1}{|}\PY{l+s+si}{\PYZob{}etape:\PYZca{}16\PYZcb{}}\PY{l+s+s1}{|}\PY{l+s+si}{\PYZob{}t:\PYZca{}12\PYZcb{}}\PY{l+s+s1}{|}\PY{l+s+si}{\PYZob{}ft:\PYZca{}12\PYZcb{}}\PY{l+s+s1}{|}\PY{l+s+s1}{\PYZsq{}}\PY{o}{.}\PY{n}{format}\PY{p}{(}\PY{n}{etape}\PY{o}{=}\PY{l+s+s1}{\PYZsq{}}\PY{l+s+s1}{Etape}\PY{l+s+s1}{\PYZsq{}}\PY{p}{,} \PY{n}{t}\PY{o}{=}\PY{l+s+s1}{\PYZsq{}}\PY{l+s+s1}{t}\PY{l+s+s1}{\PYZsq{}}\PY{p}{,} \PY{n}{ft}\PY{o}{=}\PY{l+s+s1}{\PYZsq{}}\PY{l+s+s1}{g(t)}\PY{l+s+s1}{\PYZsq{}}\PY{p}{)}\PY{p}{)}
             \PY{n}{count} \PY{o}{=} \PY{l+m+mi}{1}
             \PY{k}{while} \PY{n}{comparaison}\PY{p}{(}\PY{n}{g}\PY{p}{(}\PY{n}{x}\PY{p}{)}\PY{p}{,} \PY{n}{k}\PY{p}{)}\PY{p}{:}
                 \PY{n+nb}{print}\PY{p}{(}\PY{l+s+s1}{\PYZsq{}}\PY{l+s+s1}{|}\PY{l+s+si}{\PYZob{}etape:\PYZca{}16\PYZcb{}}\PY{l+s+s1}{|}\PY{l+s+si}{\PYZob{}t:\PYZca{}12.6f\PYZcb{}}\PY{l+s+s1}{|}\PY{l+s+si}{\PYZob{}ft:\PYZca{}12.6f\PYZcb{}}\PY{l+s+s1}{|}\PY{l+s+s1}{\PYZsq{}}\PY{o}{.}\PY{n}{format}\PY{p}{(}\PY{n}{etape}\PY{o}{=}\PY{n}{count}\PY{p}{,}\PY{n}{t}\PY{o}{=}\PY{n}{x}\PY{p}{,} \PY{n}{ft}\PY{o}{=}\PY{n}{g}\PY{p}{(}\PY{n}{x}\PY{p}{)}\PY{p}{)}\PY{p}{)}
                 \PY{n}{x} \PY{o}{+}\PY{o}{=} \PY{n}{pas}
                 \PY{n}{count} \PY{o}{+}\PY{o}{=} \PY{l+m+mi}{1}
             \PY{n+nb}{print}\PY{p}{(}\PY{l+s+s1}{\PYZsq{}}\PY{l+s+s1}{|}\PY{l+s+si}{\PYZob{}etape:\PYZca{}16\PYZcb{}}\PY{l+s+s1}{|}\PY{l+s+si}{\PYZob{}t:\PYZca{}12.6f\PYZcb{}}\PY{l+s+s1}{|}\PY{l+s+si}{\PYZob{}ft:\PYZca{}12.6f\PYZcb{}}\PY{l+s+s1}{|}\PY{l+s+s1}{\PYZsq{}}\PY{o}{.}\PY{n}{format}\PY{p}{(}\PY{n}{etape}\PY{o}{=}\PY{n}{count}\PY{p}{,}\PY{n}{t}\PY{o}{=}\PY{n}{x}\PY{p}{,} \PY{n}{ft}\PY{o}{=}\PY{n}{g}\PY{p}{(}\PY{n}{x}\PY{p}{)}\PY{p}{)}\PY{p}{)}       
             \PY{k}{return} \PY{n}{x} \PY{o}{\PYZhy{}} \PY{n}{pas}\PY{p}{,} \PY{n}{x}
\end{Verbatim}


    \begin{Verbatim}[commandchars=\\\{\}]
{\color{incolor}In [{\color{incolor}16}]:} \PY{n}{balayage}\PY{p}{(}\PY{n}{f}\PY{p}{,} \PY{l+m+mi}{8}\PY{p}{,} \PY{l+m+mi}{9}\PY{p}{,} \PY{l+m+mf}{0.1}\PY{p}{,} \PY{l+m+mi}{0}\PY{p}{)}
\end{Verbatim}


    \begin{Verbatim}[commandchars=\\\{\}]
|     Etape      |     t      |    g(t)    |
|       1        |  8.000000  |-288.000000 |
|       2        |  8.100000  |-221.870000 |
|       3        |  8.200000  |-147.520000 |
|       4        |  8.300000  | -64.710000 |
|       5        |  8.400000  | 26.800000  |

    \end{Verbatim}
\texttt{\color{outcolor}Out[{\color{outcolor}16}]:}
    
    $\displaystyle \left( 8.299999999999999, \  8.399999999999999\right)$

    

    \begin{Verbatim}[commandchars=\\\{\}]
{\color{incolor}In [{\color{incolor}23}]:} \PY{n}{balayage}\PY{p}{(}\PY{n}{f}\PY{p}{,} \PY{l+m+mf}{8.3}\PY{p}{,} \PY{l+m+mf}{8.4}\PY{p}{,} \PY{l+m+mf}{0.01}\PY{p}{,} \PY{l+m+mi}{0}\PY{p}{)}
\end{Verbatim}


    \begin{Verbatim}[commandchars=\\\{\}]
|     Etape      |     t      |    g(t)    |
|       1        |  8.300000  | -64.710000 |
|       2        |  8.310000  | -55.954460 |
|       3        |  8.320000  | -47.111680 |
|       4        |  8.330000  | -38.181420 |
|       5        |  8.340000  | -29.163440 |
|       6        |  8.350000  | -20.057500 |
|       7        |  8.360000  | -10.863360 |
|       8        |  8.370000  | -1.580780  |
|       9        |  8.380000  |  7.790480  |

    \end{Verbatim}
\texttt{\color{outcolor}Out[{\color{outcolor}23}]:}
    
    $\displaystyle \left( 8.37, \  8.379999999999999\right)$

    

    \begin{Verbatim}[commandchars=\\\{\}]
{\color{incolor}In [{\color{incolor}22}]:} \PY{n}{balayage}\PY{p}{(}\PY{n}{f}\PY{p}{,} \PY{l+m+mf}{8.36}\PY{p}{,} \PY{l+m+mf}{8.38}\PY{p}{,} \PY{l+m+mf}{0.001}\PY{p}{,} \PY{l+m+mi}{0}\PY{p}{)}
\end{Verbatim}


    \begin{Verbatim}[commandchars=\\\{\}]
|     Etape      |     t      |    g(t)    |
|       1        |  8.360000  | -10.863360 |
|       2        |  8.361000  | -9.939086  |
|       3        |  8.362000  | -9.013927  |
|       4        |  8.363000  | -8.087883  |
|       5        |  8.364000  | -7.160954  |
|       6        |  8.365000  | -6.233140  |
|       7        |  8.366000  | -5.304440  |
|       8        |  8.367000  | -4.374854  |
|       9        |  8.368000  | -3.444383  |
|       10       |  8.369000  | -2.513025  |
|       11       |  8.370000  | -1.580780  |
|       12       |  8.371000  | -0.647649  |
|       13       |  8.372000  |  0.286370  |

    \end{Verbatim}
\texttt{\color{outcolor}Out[{\color{outcolor}22}]:}
    
    $\displaystyle \left( 8.370999999999993, \  8.371999999999993\right)$

    

    \subsection{Résolution approchée par
dichotomie}\label{ruxe9solution-approchuxe9e-par-dichotomie}

    \subsubsection{Fonctions Python}\label{fonctions-python}

    \begin{Verbatim}[commandchars=\\\{\}]
{\color{incolor}In [{\color{incolor}19}]:} \PY{k}{def} \PY{n+nf}{dicho}\PY{p}{(}\PY{n}{f}\PY{p}{,}\PY{n}{a}\PY{p}{,}\PY{n}{b}\PY{p}{,}\PY{n}{e}\PY{p}{,} \PY{n}{k}\PY{p}{)}\PY{p}{:}
             \PY{l+s+sd}{\PYZdq{}\PYZdq{}\PYZdq{}valeur approchée à e près de la solution de f(x)=k}
         \PY{l+s+sd}{    sur [a,b]. On admet que l\PYZsq{}utilisateur saisit des paramètres}
         \PY{l+s+sd}{    où la dichotomie peut s\PYZsq{}appliquer. Construit une figure }
         \PY{l+s+sd}{    à chaque étape.}
         \PY{l+s+sd}{    Les valeurs de a et b affichées sont celles en sortie de boucle.\PYZdq{}\PYZdq{}\PYZdq{}}
             \PY{n}{etape} \PY{o}{=} \PY{l+m+mi}{0} \PY{c+c1}{\PYZsh{}nombre d\PYZsq{}étapes}
             \PY{k}{if} \PY{n}{f}\PY{p}{(}\PY{n}{a}\PY{p}{)} \PY{o}{\PYZlt{}}\PY{o}{=} \PY{n}{f}\PY{p}{(}\PY{n}{b}\PY{p}{)}\PY{p}{:}
                 \PY{c+c1}{\PYZsh{}croissant ou du moins passage de \PYZhy{} à +}
                 \PY{n}{croissant} \PY{o}{=} \PY{k+kc}{True}
             \PY{k}{else}\PY{p}{:}
                 \PY{n}{croissant} \PY{o}{=} \PY{k+kc}{False}
             \PY{k}{while} \PY{n}{b} \PY{o}{\PYZhy{}} \PY{n}{a} \PY{o}{\PYZgt{}} \PY{n}{e}\PY{p}{:}
                 \PY{n}{etape} \PY{o}{+}\PY{o}{=} \PY{l+m+mi}{1}
                 \PY{c+c1}{\PYZsh{}on calcule le milieu du segment [a,b]}
                 \PY{n}{m} \PY{o}{=} \PY{p}{(}\PY{n}{a}\PY{o}{+}\PY{n}{b}\PY{p}{)}\PY{o}{/}\PY{l+m+mi}{2}
                 \PY{c+c1}{\PYZsh{}figure}
                 \PY{n}{x} \PY{o}{=} \PY{n}{np}\PY{o}{.}\PY{n}{linspace}\PY{p}{(}\PY{n}{a}\PY{p}{,}\PY{n}{b}\PY{p}{,}\PY{l+m+mi}{500}\PY{p}{)}
                 \PY{n}{y} \PY{o}{=} \PY{n}{f}\PY{p}{(}\PY{n}{x}\PY{p}{)}
                 \PY{n}{plt}\PY{o}{.}\PY{n}{xlim}\PY{p}{(}\PY{n}{a}\PY{p}{,}\PY{n}{b}\PY{p}{)}
                 \PY{k}{if} \PY{n}{croissant}\PY{p}{:}
                     \PY{n}{plt}\PY{o}{.}\PY{n}{ylim}\PY{p}{(}\PY{n+nb}{max}\PY{p}{(}\PY{n}{f}\PY{p}{(}\PY{n}{a}\PY{p}{)}\PY{p}{,}\PY{o}{\PYZhy{}}\PY{l+m+mi}{50}\PY{p}{)}\PY{p}{,}\PY{n+nb}{min}\PY{p}{(}\PY{n}{f}\PY{p}{(}\PY{n}{b}\PY{p}{)}\PY{p}{,}\PY{l+m+mi}{50}\PY{p}{)}\PY{p}{)}
                 \PY{k}{else}\PY{p}{:}
                     \PY{n}{plt}\PY{o}{.}\PY{n}{ylim}\PY{p}{(}\PY{n+nb}{max}\PY{p}{(}\PY{n}{f}\PY{p}{(}\PY{n}{b}\PY{p}{)}\PY{p}{,}\PY{o}{\PYZhy{}}\PY{l+m+mi}{50}\PY{p}{)}\PY{p}{,}\PY{n+nb}{min}\PY{p}{(}\PY{n}{f}\PY{p}{(}\PY{n}{a}\PY{p}{)}\PY{p}{,}\PY{l+m+mi}{50}\PY{p}{)}\PY{p}{)}
                 \PY{n}{plt}\PY{o}{.}\PY{n}{plot}\PY{p}{(}\PY{n}{x}\PY{p}{,}\PY{n}{y}\PY{p}{,}\PY{n}{color}\PY{o}{=}\PY{l+s+s1}{\PYZsq{}}\PY{l+s+s1}{red}\PY{l+s+s1}{\PYZsq{}}\PY{p}{)}
                 \PY{n}{plt}\PY{o}{.}\PY{n}{grid}\PY{p}{(}\PY{k+kc}{True}\PY{p}{)}
                 \PY{n}{plt}\PY{o}{.}\PY{n}{axhline}\PY{p}{(}\PY{n}{k}\PY{p}{)}
                 \PY{n}{plt}\PY{o}{.}\PY{n}{title}\PY{p}{(}\PY{l+s+s1}{\PYZsq{}}\PY{l+s+s1}{a=}\PY{l+s+si}{\PYZpc{}.4f}\PY{l+s+s1}{ et m=}\PY{l+s+si}{\PYZpc{}.4f}\PY{l+s+s1}{ et b=}\PY{l+s+si}{\PYZpc{}.4f}\PY{l+s+s1}{\PYZsq{}}\PY{o}{\PYZpc{}}\PY{p}{(}\PY{n}{a}\PY{p}{,}\PY{n}{m}\PY{p}{,}\PY{n}{b}\PY{p}{)}\PY{p}{)}
                 \PY{n}{plt}\PY{o}{.}\PY{n}{show}\PY{p}{(}\PY{p}{)}
                 \PY{c+c1}{\PYZsh{}fin de la figure}
                 \PY{n}{s} \PY{o}{=} \PY{p}{(}\PY{n}{f}\PY{p}{(}\PY{n}{m}\PY{p}{)} \PY{o}{\PYZhy{}} \PY{n}{k}\PY{p}{)}\PY{o}{*}\PY{p}{(}\PY{n}{f}\PY{p}{(}\PY{n}{a}\PY{p}{)} \PY{o}{\PYZhy{}} \PY{n}{k}\PY{p}{)}
                 \PY{c+c1}{\PYZsh{}si f(m) \PYZhy{} k  et f(a) \PYZhy{} k sont de meme signe, f(x)=k dans ]m,b[}
                 \PY{k}{if} \PY{n}{s} \PY{o}{\PYZgt{}} \PY{l+m+mi}{0}\PY{p}{:}
                     \PY{n}{a} \PY{o}{=} \PY{n}{m}
                 \PY{c+c1}{\PYZsh{}si f(m) \PYZhy{} k  et f(a) \PYZhy{} k sont de signes opposés, f(x)=k dans ]a,m]}
                 \PY{k}{else}\PY{p}{:}
                     \PY{n}{b} \PY{o}{=} \PY{n}{m}       
             \PY{k}{return} \PY{n}{a}\PY{p}{,} \PY{n}{b}\PY{p}{,} \PY{n}{etape}
         
         \PY{k}{def} \PY{n+nf}{dicho\PYZus{}tab}\PY{p}{(}\PY{n}{f}\PY{p}{,}\PY{n}{a}\PY{p}{,}\PY{n}{b}\PY{p}{,}\PY{n}{e}\PY{p}{,} \PY{n}{k}\PY{p}{)}\PY{p}{:}
             \PY{l+s+sd}{\PYZdq{}\PYZdq{}\PYZdq{}valeur approchée à e près de la solution de f(x)=k}
         \PY{l+s+sd}{    sur [a,b]. On admet que l\PYZsq{}utilisateur saisit des paramètres}
         \PY{l+s+sd}{    où la dichotomie peut s\PYZsq{}appliquer. }
         \PY{l+s+sd}{    Ne retourne rien mais remplit un tableau avec les valeurs}
         \PY{l+s+sd}{    de a, b et m aux differentes etapes.}
         \PY{l+s+sd}{    Les valeurs de a et b affichées sont celles en sortie de boucle\PYZdq{}\PYZdq{}\PYZdq{}}
             \PY{n}{count} \PY{o}{=} \PY{l+m+mi}{0} \PY{c+c1}{\PYZsh{}nombre d\PYZsq{}étapes}
             \PY{k}{if} \PY{n}{f}\PY{p}{(}\PY{n}{a}\PY{p}{)} \PY{o}{\PYZlt{}}\PY{o}{=} \PY{n}{f}\PY{p}{(}\PY{n}{b}\PY{p}{)}\PY{p}{:}
                 \PY{c+c1}{\PYZsh{}croissant ou du moins passage de \PYZhy{} à +}
                 \PY{n}{croissant} \PY{o}{=} \PY{k+kc}{True}
             \PY{k}{else}\PY{p}{:}
                 \PY{n}{croissant} \PY{o}{=} \PY{k+kc}{False}
             \PY{c+c1}{\PYZsh{}en\PYZhy{}tete du tableau}
             \PY{n+nb}{print}\PY{p}{(}\PY{l+s+s1}{\PYZsq{}}\PY{l+s+s1}{|}\PY{l+s+si}{\PYZob{}etape:\PYZca{}16\PYZcb{}}\PY{l+s+s1}{|}\PY{l+s+si}{\PYZob{}median:\PYZca{}12\PYZcb{}}\PY{l+s+s1}{|}\PY{l+s+si}{\PYZob{}test:\PYZca{}10\PYZcb{}}\PY{l+s+s1}{|}\PY{l+s+si}{\PYZob{}binf:\PYZca{}12\PYZcb{}}\PY{l+s+s1}{|}\PY{l+s+si}{\PYZob{}bsup:\PYZca{}12\PYZcb{}}\PY{l+s+s1}{|}\PY{l+s+s1}{\PYZsq{}}\PY{o}{.}\PY{n}{format}\PY{p}{(}\PY{n}{etape}\PY{o}{=}\PY{l+s+s1}{\PYZsq{}}\PY{l+s+s1}{Etape}\PY{l+s+s1}{\PYZsq{}}\PY{p}{,}
                                                                              \PY{n}{binf}\PY{o}{=}\PY{l+s+s1}{\PYZsq{}}\PY{l+s+s1}{a}\PY{l+s+s1}{\PYZsq{}}\PY{p}{,} \PY{n}{bsup}\PY{o}{=}\PY{l+s+s1}{\PYZsq{}}\PY{l+s+s1}{b}\PY{l+s+s1}{\PYZsq{}}\PY{p}{,} \PY{n}{median}\PY{o}{=}\PY{l+s+s1}{\PYZsq{}}\PY{l+s+s1}{m}\PY{l+s+s1}{\PYZsq{}}\PY{p}{,}
                                                                              \PY{n}{test}\PY{o}{=}\PY{l+s+s1}{\PYZsq{}}\PY{l+s+s1}{Choix ?}\PY{l+s+s1}{\PYZsq{}}\PY{p}{)}\PY{p}{)}
             \PY{c+c1}{\PYZsh{}première ligne du tableau}
             \PY{c+c1}{\PYZsh{}remplissage de la ligne du tableau}
             \PY{n+nb}{print}\PY{p}{(}\PY{l+s+s1}{\PYZsq{}}\PY{l+s+s1}{|}\PY{l+s+si}{\PYZob{}etape:\PYZca{}16\PYZcb{}}\PY{l+s+s1}{|}\PY{l+s+si}{\PYZob{}median:\PYZca{}12\PYZcb{}}\PY{l+s+s1}{|}\PY{l+s+si}{\PYZob{}test:\PYZca{}10\PYZcb{}}\PY{l+s+s1}{|}\PY{l+s+si}{\PYZob{}binf:\PYZca{}12\PYZcb{}}\PY{l+s+s1}{|}\PY{l+s+si}{\PYZob{}bsup:\PYZca{}12\PYZcb{}}\PY{l+s+s1}{|}\PY{l+s+s1}{\PYZsq{}}\PY{o}{.}\PY{n}{format}\PY{p}{(}\PY{n}{etape}\PY{o}{=}\PY{l+s+s1}{\PYZsq{}}\PY{l+s+s1}{initialisation}\PY{l+s+s1}{\PYZsq{}}\PY{p}{,}
                                                                              \PY{n}{binf}\PY{o}{=}\PY{n}{a}\PY{p}{,} \PY{n}{bsup}\PY{o}{=}\PY{n}{b}\PY{p}{,} \PY{n}{median}\PY{o}{=}\PY{n+nb}{str}\PY{p}{(}\PY{k+kc}{None}\PY{p}{)}\PY{p}{,}
                                                                              \PY{n}{test}\PY{o}{=}\PY{n+nb}{str}\PY{p}{(}\PY{k+kc}{None}\PY{p}{)}\PY{p}{)}\PY{p}{)}
             \PY{k}{while} \PY{n}{b} \PY{o}{\PYZhy{}} \PY{n}{a} \PY{o}{\PYZgt{}} \PY{n}{e}\PY{p}{:}
                 \PY{n}{count} \PY{o}{+}\PY{o}{=} \PY{l+m+mi}{1}
                 \PY{c+c1}{\PYZsh{}on calcule le milieu du segment [a,b]}
                 \PY{n}{m} \PY{o}{=} \PY{p}{(}\PY{n}{a}\PY{o}{+}\PY{n}{b}\PY{p}{)}\PY{o}{/}\PY{l+m+mi}{2}
                 \PY{n}{s} \PY{o}{=} \PY{p}{(}\PY{n}{f}\PY{p}{(}\PY{n}{m}\PY{p}{)} \PY{o}{\PYZhy{}} \PY{n}{k}\PY{p}{)}\PY{o}{*}\PY{p}{(}\PY{n}{f}\PY{p}{(}\PY{n}{a}\PY{p}{)} \PY{o}{\PYZhy{}} \PY{n}{k}\PY{p}{)}
                 \PY{c+c1}{\PYZsh{}si f(m) \PYZhy{} k  et f(a) \PYZhy{} k sont de meme signe, f(x)=k dans ]m,b[}
                 \PY{k}{if} \PY{n}{s} \PY{o}{\PYZgt{}} \PY{l+m+mi}{0}\PY{p}{:}
                     \PY{n}{a} \PY{o}{=} \PY{n}{m}
                 \PY{c+c1}{\PYZsh{}si f(m) \PYZhy{} k  et f(a) \PYZhy{} k sont de signes opposés, f(x)=k dans ]a,m]}
                 \PY{k}{else}\PY{p}{:}
                     \PY{n}{b} \PY{o}{=} \PY{n}{m}
                 \PY{c+c1}{\PYZsh{}remplissage de la ligne du tableau}
                 \PY{n+nb}{print}\PY{p}{(}\PY{l+s+s1}{\PYZsq{}}\PY{l+s+s1}{|}\PY{l+s+si}{\PYZob{}etape:\PYZca{}16\PYZcb{}}\PY{l+s+s1}{|}\PY{l+s+si}{\PYZob{}median:\PYZca{}12\PYZcb{}}\PY{l+s+s1}{|}\PY{l+s+si}{\PYZob{}test:\PYZca{}10\PYZcb{}}\PY{l+s+s1}{|}\PY{l+s+si}{\PYZob{}binf:\PYZca{}12\PYZcb{}}\PY{l+s+s1}{|}\PY{l+s+si}{\PYZob{}bsup:\PYZca{}12\PYZcb{}}\PY{l+s+s1}{|}\PY{l+s+s1}{\PYZsq{}}\PY{o}{.}\PY{n}{format}\PY{p}{(}\PY{n}{etape}\PY{o}{=}\PY{n}{count}\PY{p}{,}
                                                                              \PY{n}{binf}\PY{o}{=}\PY{n}{a}\PY{p}{,} \PY{n}{bsup}\PY{o}{=}\PY{n}{b}\PY{p}{,} \PY{n}{median}\PY{o}{=}\PY{n}{m}\PY{p}{,}
                                                                              \PY{n}{test}\PY{o}{=} \PY{l+s+s1}{\PYZsq{}}\PY{l+s+s1}{gauche}\PY{l+s+s1}{\PYZsq{}} \PY{k}{if} \PY{n}{b} \PY{o}{==} \PY{n}{m} \PY{k}{else} \PY{l+s+s1}{\PYZsq{}}\PY{l+s+s1}{droite}\PY{l+s+s1}{\PYZsq{}}\PY{p}{)}\PY{p}{)}   
\end{Verbatim}


    \subsection{\texorpdfstring{Résolution par dichotomie de l'équation
\(f(x)=0\) dans l'intervalle
\([8;9]\)}{Résolution par dichotomie de l'équation f(x)=0 dans l'intervalle {[}8;9{]}}}\label{ruxe9solution-par-dichotomie-de-luxe9quation-fx0-dans-lintervalle-89}

    \subsubsection{\texorpdfstring{D'abord on s'arrete lorsque l'amplitude
de l'intervalle \([a,b]\) est inférieure ou égale à
\(0,02\)}{D'abord on s'arrete lorsque l'amplitude de l'intervalle {[}a,b{]} est inférieure ou égale à 0,02}}\label{dabord-on-sarrete-lorsque-lamplitude-de-lintervalle-ab-est-infuxe9rieure-ou-uxe9gale-uxe0-002}

    \begin{Verbatim}[commandchars=\\\{\}]
{\color{incolor}In [{\color{incolor}29}]:} \PY{n}{dicho\PYZus{}tab}\PY{p}{(}\PY{n}{f}\PY{p}{,} \PY{l+m+mi}{8}\PY{p}{,} \PY{l+m+mi}{9}\PY{p}{,} \PY{l+m+mf}{0.02}\PY{p}{,} \PY{l+m+mi}{0}\PY{p}{)}
\end{Verbatim}


    \begin{Verbatim}[commandchars=\\\{\}]
|     Etape      |     m      | Choix ?  |     a      |     b      |
| initialisation |    None    |   None   |     8      |     9      |
|       1        |    8.5     |  gauche  |     8      |    8.5     |
|       2        |    8.25    |  droite  |    8.25    |    8.5     |
|       3        |   8.375    |  gauche  |    8.25    |   8.375    |
|       4        |   8.3125   |  droite  |   8.3125   |   8.375    |
|       5        |  8.34375   |  droite  |  8.34375   |   8.375    |
|       6        |  8.359375  |  droite  |  8.359375  |   8.375    |

    \end{Verbatim}

    \begin{Verbatim}[commandchars=\\\{\}]
{\color{incolor}In [{\color{incolor}30}]:} \PY{n}{dicho}\PY{p}{(}\PY{n}{f}\PY{p}{,} \PY{l+m+mi}{8}\PY{p}{,} \PY{l+m+mi}{9}\PY{p}{,} \PY{l+m+mf}{0.02}\PY{p}{,} \PY{l+m+mi}{0}\PY{p}{)}
\end{Verbatim}


    \begin{center}
    \adjustimage{max size={0.9\linewidth}{0.9\paperheight}}{output_28_0.png}
    \end{center}
    { \hspace*{\fill} \\}
    
    \begin{center}
    \adjustimage{max size={0.9\linewidth}{0.9\paperheight}}{output_28_1.png}
    \end{center}
    { \hspace*{\fill} \\}
    
    \begin{center}
    \adjustimage{max size={0.9\linewidth}{0.9\paperheight}}{output_28_2.png}
    \end{center}
    { \hspace*{\fill} \\}
    
    \begin{center}
    \adjustimage{max size={0.9\linewidth}{0.9\paperheight}}{output_28_3.png}
    \end{center}
    { \hspace*{\fill} \\}
    
    \begin{center}
    \adjustimage{max size={0.9\linewidth}{0.9\paperheight}}{output_28_4.png}
    \end{center}
    { \hspace*{\fill} \\}
    
    \begin{center}
    \adjustimage{max size={0.9\linewidth}{0.9\paperheight}}{output_28_5.png}
    \end{center}
    { \hspace*{\fill} \\}
    \texttt{\color{outcolor}Out[{\color{outcolor}30}]:}
    
    $\displaystyle \left( 8.359375, \  8.375, \  6\right)$

    

    \subsubsection{\texorpdfstring{Ensuite on s'arrete lorsque l'amplitude
de l'intervalle \([a,b]\) est inférieure ou égale à
\(0,002\)}{Ensuite on s'arrete lorsque l'amplitude de l'intervalle {[}a,b{]} est inférieure ou égale à 0,002}}\label{ensuite-on-sarrete-lorsque-lamplitude-de-lintervalle-ab-est-infuxe9rieure-ou-uxe9gale-uxe0-0002}

    \begin{Verbatim}[commandchars=\\\{\}]
{\color{incolor}In [{\color{incolor}31}]:} \PY{n}{dicho\PYZus{}tab}\PY{p}{(}\PY{n}{f}\PY{p}{,} \PY{l+m+mi}{8}\PY{p}{,} \PY{l+m+mi}{9}\PY{p}{,} \PY{l+m+mf}{0.002}\PY{p}{,} \PY{l+m+mi}{0}\PY{p}{)}
\end{Verbatim}


    \begin{Verbatim}[commandchars=\\\{\}]
|     Etape      |     m      | Choix ?  |     a      |     b      |
| initialisation |    None    |   None   |     8      |     9      |
|       1        |    8.5     |  gauche  |     8      |    8.5     |
|       2        |    8.25    |  droite  |    8.25    |    8.5     |
|       3        |   8.375    |  gauche  |    8.25    |   8.375    |
|       4        |   8.3125   |  droite  |   8.3125   |   8.375    |
|       5        |  8.34375   |  droite  |  8.34375   |   8.375    |
|       6        |  8.359375  |  droite  |  8.359375  |   8.375    |
|       7        | 8.3671875  |  droite  | 8.3671875  |   8.375    |
|       8        | 8.37109375 |  droite  | 8.37109375 |   8.375    |
|       9        |8.373046875 |  gauche  | 8.37109375 |8.373046875 |

    \end{Verbatim}

    \begin{Verbatim}[commandchars=\\\{\}]
{\color{incolor}In [{\color{incolor}32}]:} \PY{n}{dicho}\PY{p}{(}\PY{n}{f}\PY{p}{,} \PY{l+m+mi}{8}\PY{p}{,} \PY{l+m+mi}{9}\PY{p}{,} \PY{l+m+mf}{0.002}\PY{p}{,} \PY{l+m+mi}{0}\PY{p}{)}
\end{Verbatim}


    \begin{center}
    \adjustimage{max size={0.9\linewidth}{0.9\paperheight}}{output_31_0.png}
    \end{center}
    { \hspace*{\fill} \\}
    
    \begin{center}
    \adjustimage{max size={0.9\linewidth}{0.9\paperheight}}{output_31_1.png}
    \end{center}
    { \hspace*{\fill} \\}
    
    \begin{center}
    \adjustimage{max size={0.9\linewidth}{0.9\paperheight}}{output_31_2.png}
    \end{center}
    { \hspace*{\fill} \\}
    
    \begin{center}
    \adjustimage{max size={0.9\linewidth}{0.9\paperheight}}{output_31_3.png}
    \end{center}
    { \hspace*{\fill} \\}
    
    \begin{center}
    \adjustimage{max size={0.9\linewidth}{0.9\paperheight}}{output_31_4.png}
    \end{center}
    { \hspace*{\fill} \\}
    
    \begin{center}
    \adjustimage{max size={0.9\linewidth}{0.9\paperheight}}{output_31_5.png}
    \end{center}
    { \hspace*{\fill} \\}
    
    \begin{center}
    \adjustimage{max size={0.9\linewidth}{0.9\paperheight}}{output_31_6.png}
    \end{center}
    { \hspace*{\fill} \\}
    
    \begin{center}
    \adjustimage{max size={0.9\linewidth}{0.9\paperheight}}{output_31_7.png}
    \end{center}
    { \hspace*{\fill} \\}
    
    \begin{center}
    \adjustimage{max size={0.9\linewidth}{0.9\paperheight}}{output_31_8.png}
    \end{center}
    { \hspace*{\fill} \\}
    \texttt{\color{outcolor}Out[{\color{outcolor}32}]:}
    
    $\displaystyle \left( 8.37109375, \  8.373046875, \  9\right)$

    

    \begin{Verbatim}[commandchars=\\\{\}]
{\color{incolor}In [{\color{incolor}28}]:} \PY{n}{n} \PY{o}{=} \PY{l+m+mi}{0}
         \PY{k}{while} \PY{l+m+mi}{1}\PY{o}{/}\PY{l+m+mi}{2}\PY{o}{*}\PY{o}{*}\PY{n}{n} \PY{o}{\PYZgt{}} \PY{l+m+mf}{0.001}\PY{p}{:}
             \PY{n}{n} \PY{o}{+}\PY{o}{=} \PY{l+m+mi}{1}
         \PY{n+nb}{print}\PY{p}{(}\PY{n}{n}\PY{p}{)}
\end{Verbatim}


    \begin{Verbatim}[commandchars=\\\{\}]
10

    \end{Verbatim}

    L'intervalle de départ a pour amplitude 1 (\(a_0=-2\) et \(b_0=-1\)) A
chaque étape l'amplitude de l'intervalle de recherche est divisée par 2
Au bout de \(n\) étapes, l'amplitude de la zone de recherche est de
\(\frac{b_{0}-a_{0}}{2^n}\) soit \(\frac{1}{2^n}\) ici. Avec la
calculatrice (voir ci-dessus ou le logarithme népérien) on peut vérifier
que le plus petit entier \(n\) tel que \(\frac{1}{2^n} \leqslant 0,001\)
est \(n=10\)

    \subsection{Exercice 3 TVI et dichotomie fiche d'exercices
2018}\label{exercice-3-tvi-et-dichotomie-fiche-dexercices-2018}

    \begin{Verbatim}[commandchars=\\\{\}]
{\color{incolor}In [{\color{incolor}22}]:} \PY{n}{dicho\PYZus{}tab}\PY{p}{(}\PY{k}{lambda} \PY{n}{x} \PY{p}{:} \PY{l+m+mi}{2} \PY{o}{*} \PY{n}{x} \PY{o}{*}\PY{o}{*} \PY{l+m+mi}{3} \PY{o}{+} \PY{l+m+mi}{6} \PY{o}{*} \PY{n}{x} \PY{o}{*}\PY{o}{*} \PY{l+m+mi}{2} \PY{o}{\PYZhy{}} \PY{l+m+mi}{1}\PY{p}{,} \PY{l+m+mi}{0}\PY{p}{,} \PY{l+m+mf}{0.5}\PY{p}{,} \PY{l+m+mf}{0.01}\PY{p}{,} \PY{l+m+mi}{0}\PY{p}{)}
\end{Verbatim}


    \begin{Verbatim}[commandchars=\\\{\}]
|     Etape      |     m      | Choix ?  |     a      |     b      |
| initialisation |    None    |   None   |     0      |    0.5     |
|       1        |    0.25    |  droite  |    0.25    |    0.5     |
|       2        |   0.375    |  droite  |   0.375    |    0.5     |
|       3        |   0.4375   |  gauche  |   0.375    |   0.4375   |
|       4        |  0.40625   |  gauche  |   0.375    |  0.40625   |
|       5        |  0.390625  |  gauche  |   0.375    |  0.390625  |
|       6        | 0.3828125  |  droite  | 0.3828125  |  0.390625  |

    \end{Verbatim}

    \begin{Verbatim}[commandchars=\\\{\}]
{\color{incolor}In [{\color{incolor}23}]:} \PY{n}{dicho}\PY{p}{(}\PY{k}{lambda} \PY{n}{x} \PY{p}{:} \PY{l+m+mi}{2} \PY{o}{*} \PY{n}{x} \PY{o}{*}\PY{o}{*} \PY{l+m+mi}{3} \PY{o}{+} \PY{l+m+mi}{6} \PY{o}{*} \PY{n}{x} \PY{o}{*}\PY{o}{*} \PY{l+m+mi}{2} \PY{o}{\PYZhy{}} \PY{l+m+mi}{1}\PY{p}{,} \PY{l+m+mi}{0}\PY{p}{,} \PY{l+m+mf}{0.5}\PY{p}{,} \PY{l+m+mf}{0.01}\PY{p}{,} \PY{l+m+mi}{0}\PY{p}{)}
\end{Verbatim}


    \begin{center}
    \adjustimage{max size={0.9\linewidth}{0.9\paperheight}}{output_36_0.png}
    \end{center}
    { \hspace*{\fill} \\}
    
    \begin{center}
    \adjustimage{max size={0.9\linewidth}{0.9\paperheight}}{output_36_1.png}
    \end{center}
    { \hspace*{\fill} \\}
    
    \begin{center}
    \adjustimage{max size={0.9\linewidth}{0.9\paperheight}}{output_36_2.png}
    \end{center}
    { \hspace*{\fill} \\}
    
    \begin{center}
    \adjustimage{max size={0.9\linewidth}{0.9\paperheight}}{output_36_3.png}
    \end{center}
    { \hspace*{\fill} \\}
    
    \begin{center}
    \adjustimage{max size={0.9\linewidth}{0.9\paperheight}}{output_36_4.png}
    \end{center}
    { \hspace*{\fill} \\}
    
    \begin{center}
    \adjustimage{max size={0.9\linewidth}{0.9\paperheight}}{output_36_5.png}
    \end{center}
    { \hspace*{\fill} \\}
    \texttt{\color{outcolor}Out[{\color{outcolor}23}]:}
    
    $$\left ( 0.3828125, \quad 0.390625, \quad 6\right )$$

    


    % Add a bibliography block to the postdoc
    
    
    
    \end{document}
